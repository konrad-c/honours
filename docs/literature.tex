\documentclass{article}
\usepackage[utf8]{inputenc}
\usepackage{enumitem}

\title{Literature Review: Synthesising emotion-driven images}
\author{Konrad Cybulski}
\date{May 2019}

\usepackage[numbers]{natbib}
\usepackage{graphicx}

\begin{document}
	
\maketitle

\section{Aims and Scope}

The primary focus of the proposed research is to produce an image emotion classifier, and to further leverage it in the process of emotion-driven image synthesis.
The two core themes explored are image generation techniques, and processes for classifying and generating emotional content.
As a result this literature review aims to explore and understand the progression of knowledge in the fields of emotion representation and classification, computational image synthesis, and affective content synthesis.
All of which represent the multiple facets of the proposed body of research with which an investigation into the use of image emotion classifiers for emotion-driven image generation will be explored.

\section{Emotion representation and classification}

The field of emotion classification has surged in recent years given a popularity and rising interest in facial emotion recognition.
Other research has focused on exploring ways in which emotion can be recognised in images, text, and more abstract content.
Underlying these two areas of emotion recognition is the methods for quantitatively representing emotion in both a meaningful, and accurate way.
This section will discuss the evolution of these aspects of emotion classification, in addition to their commonalities, and differences.

The computational recognition of emotion has been explored in countless studies and projects, in both the context of image emotion recognition \citep{machajdik2010affective,zhao2014exploring,kim2018building} and facial emotion classification \citep{mollahosseini2016going}.
However a core component and key difference between a large number of such bodies of research is the method by which emotion is represented.
Two of the most common representations represent a discrete, and continuous approach.
Discrete emotion representations generally involve the categorisation of emotion to a series of labels.
Discrete approaches represent the method used in a large number of papers \citep{machajdik2010affective,ali2017emotional,wangarttalk,mohammad2018wikiart} however the number of emotion labels explored varies greatly.
An example of the size of emotion label subsets used in studies of image emotion classification using such a discrete model are 7 \citep{ali2017emotional}, 8 \citep{machajdik2010affective}, 11 \citep{wangarttalk}, and even 19 \citep{mohammad2018wikiart}.
Due to this lack of consistency in emotion classification targets, such studies often resort to performing similar data gathering and classifier training methods.
Despite the consequences of inconsistency in the emotion labels chosen, due to the simplicity of discrete categorical emotion assignment, data gathering can be performed with ease compared to continuous methods of emotion representation.
In order to create large image datasets of labelled images can be completed with greater ease when the number of such labels is reduced.
However there remains difficulty in labelling images with respect to emotion given the inherent variation in the emotion felt by someone when both viewing an abstract image, or accurately determining the emotion expressed by a facial expression.

The aforementioned difficulty associated with labelling images according to their respective emotional content is accentuated with the introduction of a dimensionally continuous representation of emotion.
While proposed representations vary, the most recognised basis for many continuous models is the circumplex model of emotion \citep{russell1980circumplex}.
The circumplex model of emotion introduced by \citet{russell1980circumplex} asserts that emotion can be measured in terms of two continuous variables: valence, and arousal.
Valence measures the positivity or negativity associated with an emotion; and arousal measures the excitement associated with it.
For example, the discrete emotion label of \textbf{happiness} could be represented in the continuous valence-arousal (VA) space with high-valence, medium-arousal; while the label \textbf{depressed} would translate to low-valence, low-arousal; and \textbf{relaxed} being medium-valence, low-arousal.
While this continuous dimensional model allows for greater continuity in measuring emotions, it is not without fault \citep{larsen1992promises}.
Despite contention, particularly with regard to its limitations, the circumplex model has been used extensively in the field of psychology, and in computational emotion classification to a limited extent \citep{zhao2016predicting}.




\section{Computational image synthesis}

\section{Affective content synthesis}

\section{Conclusion}

\bibliographystyle{apalike}
\bibliography{references}

\end{document}
