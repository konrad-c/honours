\documentclass{article}
\usepackage[utf8]{inputenc}
\usepackage{enumitem}

\title{Literature Review: Synthesising emotion-driven images}
\author{Konrad Cybulski}
\date{May 2019}

\usepackage[numbers]{natbib}
\usepackage{graphicx}

\begin{document}
	
\maketitle

\tableofcontents

\pagebreak

\section{Aims and Scope}

The primary focus of the proposed research is to produce an image emotion classifier, and to further leverage it in the process of emotion-driven image synthesis.
The two core themes explored are image generation techniques, and processes for classifying and generating emotional content.
As a result this literature review aims to explore and understand the progression of knowledge in the fields of emotion representation and classification, computational image synthesis, and affective content synthesis.
All of which represent the multiple facets of the proposed body of research with which an investigation into the use of image emotion classifiers for emotion-driven image generation will be explored.


\section{Emotion representation}

The field of emotion classification has surged in recent years given a popularity and rising interest in facial emotion recognition.
Other research has focused on exploring ways in which emotion can be recognised in images, text, and more abstract content.
Underlying these two areas of emotion recognition is the methods for quantitatively representing emotion in both a meaningful, and accurate way.
This section will discuss the evolution of these aspects of emotion classification, in addition to their commonalities, and differences.

The computational recognition of emotion has been explored in countless studies and projects, in both the context of image emotion recognition \citep{machajdik2010affective,zhao2014exploring,kim2018building} and facial emotion classification \citep{mollahosseini2016going}.
However a core component and key difference between a large number of such bodies of research is the method by which emotion is represented.
Two of the most common representations represent a discrete, and continuous approach.


\subsection{Categorical}

Discrete emotion representations generally involve the categorisation of emotion to a series of labels.
Discrete approaches represent the method used in a large number of papers \citep{machajdik2010affective,ali2017emotional,wangarttalk,mohammad2018wikiart} however the number of emotion labels explored varies greatly.
An example of the size of emotion label subsets used in studies of image emotion classification using such a discrete model are 7 \citep{ali2017emotional}, 8 \citep{machajdik2010affective}, 11 \citep{wangarttalk}, and even 19 \citep{mohammad2018wikiart}.
Due to this lack of consistency in emotion classification targets, such studies often repeat similar data gathering and classifier training methods again for their own use.
Despite the consequences of inconsistency in the emotion labels chosen, due to the simplicity of discrete categorical emotion assignment, data gathering can be performed with ease compared to continuous methods of emotion representation.
In order to create large image datasets of labelled images can be completed with greater ease when the number of such labels is reduced.
However there remains difficulty in labelling images with respect to emotion given the inherent variation in the emotion felt by someone when both viewing an abstract image, or accurately determining the emotion expressed by a facial expression.

The categorical representation is the simplest method for emotion classification, due not only to the consequent ease with which data can be gathered, but due to the extensive research surrounding methods for computational classification of content.
Many techniques for prediction, both for regression and classification rely on large datasets, something more easily available and more accurate for categorical data.
Research in psychology has understood the increased error associated with continuous measurement tasked performed by humans in comparison to categorical classification \citep{harnad2003categorical}.
With the human brain's attempt to sort perceived objects and situations into learned categories, it has been shown to warp continuous variables and scales to do so.
As a result there exists an increased error in human measurement of continuous, compared to categorical variables.


\subsection{Continuous dimensional}

The aforementioned difficulty associated with labelling images according to their respective emotional content is accentuated with the introduction of a dimensionally continuous representation of emotion.
While proposed representations vary, the most recognised basis for many continuous models is the circumplex model of emotion \citep{russell1980circumplex}.
The circumplex model of emotion introduced by \citet{russell1980circumplex} asserts that emotion can be measured in terms of two continuous variables: valence, and arousal.
Valence measures the positivity or negativity associated with an emotion; and arousal measures the excitement associated with it.
For example, the discrete emotion label of \textbf{happiness} could be represented in the continuous valence-arousal (VA) space with high-valence, medium-arousal; while the label \textbf{depressed} would translate to low-valence, low-arousal; and \textbf{relaxed} being medium-valence, low-arousal.
While this continuous dimensional model allows for greater continuity in measuring emotions, it is not without fault \citep{larsen1992promises}.
The primary limitations of such a model involve it's likelihood of misinterpretation particularly when considering the labels of each dimension, and any information loss arising from reducing more complex emotions to a two-dimensional space.
One way in which the circumplex model of affect introduced by \citet{russell1980circumplex} has been extended to address its weaknesses is through the introduction of a third-dimension: Dominance, the amount of control associated with an emotion.
Despite its limitations, the circumplex model has been used extensively in the field of psychology \citep{bradley1994measuring,warriner2013norms}, and in computational emotion classification to a limited extent \citep{zhao2016predicting}.

While both models for the representation of emotion have trade-offs, both suffer from issues related to cross-cultural differences in emotion expression and recognition.
Cultures inherently differ with respect to how emotions are both felt and expressed \citep{markus1991culture}; this is becomes increasingly evident when attempting to classify the affective emotion embodied and expressed by more abstract content such as imagery and sound.
The popular image-emotion dataset used for classification known as the International Affective Picture System (IAPS) was shown to have a significantly different valence-arousal assignment for up to 31.74\% of images between Chinese and American young adults \citep{huang2015affective}.
Valence-arousal values assigned to images often vary when considering a group of people.
To address this potential for error in reporting the subjective emotion imparted by content, the self-assessment manikin (SAM) was introduced \citep{lang1980behavioral}.
SAM, as can be seen in Figure \ref{fig:SAM}, was developed to aid in the evaluation of emotion, particularly its translation into the commonly used three dimensions of the circumplex model of affect: valence, arousal, dominance.
It has proved itself as more accurate and effective than other methods of emotion self-assessment while being less complex \citep{bradley1994measuring}.

\begin{figure}[h!]
	\centering
	\includegraphics[width=0.75\textwidth]{images/self-assessment-manikin.png}
	\caption{The self-assessment manikin: a guide for reporting individual emotional affect in the three dimensions valence (top), arousal (middle), and dominance (bottom) as introduced by \citet{lang1980behavioral}}
	\label{fig:SAM}
\end{figure}

While both discrete and continuous methods of representing emotion have their respective benefits and trade-offs, the choice to use one over another often depends on the way in which the data gathering process for content emotion classification is performed.
These two representations of emotion can be translated between, since all categorical emotions can be converted to the continuous circumplex model, and the inverse.
However due to the higher complexity nature of the continuous model, even with the use of SAM, dimensional representations are more difficult to gather on a large scale in comparison to categorical emotions.


\section{Emotion classification}

Emotion classification has been researched in depth particularly with respect to two sub-domains: facial emotion recognition, and content emotion classification (image, text, sound).
Facial emotion recognition has been more heavily researched in this field when compared to content emotion classification.
While the applications of content emotion classifications seem to be few in comparison with those of facial recognition, the field has widely been explored in the context of text emotion classification, with image and sound emotion classification trailing behind.

Emotion classification has itself been a domain built upon the foundation of image classification techniques.
From hand-crafted feature decomposition, to neural network architectures, the emotional content of images and text rely heavily on findings in other domains such as image object classification and text sentiment analysis.

\subsection{Image object classification}

As the basis for many further computer vision applications, image object classification has been a primary focus of research in the field of machine learning.
The techniques associated with such tasks have for the most part in recent years involved the use of deep convolutional neural networks, which have proven to be state-of-the-art classifiers, with several high performing architectures such as ResNet, AlexNet, and Inception being used as staple image classifiers in other domains \citep{pan2009survey}.
While neural networks and their use date back to the 1980s, it was only as computing power sufficiently increased in the 21st century that neural network architectures could be viably trained, and used \citep{rawat2017deep}.

Up until such time as neural network architectures could be fully leveraged, the majority of image classification tasks lent on manually deriving image features for use in other more traditional classification models (regression, support vector machines, etc.).
As a result the limiting factor of such classifiers were predictors derived from extracted image features.



\subsection{Facial emotion recognition}

============================================


============================================

Why is facial emotion recognition used and is of importance?
How is it done (methods)? i.e. neural networks vs feature decomposition vs computer vision etc?
Any current issues
============================================


============================================

Humans are known to be more accurate in classifying facial emotion of another when they are culturally similar due to their in-group advantage.
This was also found to be the case with computational facial emotion classifiers when trained more heavily on one culture over another \citep{dailey2010evidence}.

\subsubsection{Neural networks} \label{sec:neural-network-overview}

Neural networks have become extremely popular in various domains of both prediction, and content generation.
Before proceeding with further discussion about applications of more complex neural network architectures such as convolutional neural networks (CNN) and generative adversarial networks

\subsection{Transfer learning}


\subsection{Image emotion classification}

============================================


============================================

Brief history, applications, why it's used

Methods: feature decomposition vs CV vs neural networks

Existing problems and shortcomings of current methods.
============================================


============================================

The area of image emotion and sentiment classification has been explored in a number of ways, primarily through image feature analysis derived from art and psychological factors \citep{machajdik2010affective}; and more recently using techniques such as deep neural networks \citep{chen2015learning, kim2018building}.
Feature extraction and analysis has been used for various applications such as measuring aesthetic appeal \citep{den2010using,den2010comparing,den2011evolving} and as an emotional feature vector for sentiment classification \citep{machajdik2010affective}.
Due to the artistic and psychological underpinnings used by \citet{machajdik2010affective}, the low-level features extracted from images can be understood at a high level.
The relationship between an image's emotion and its core artistic components such as balance, harmony, and variety was further explored by \citet{zhao2014exploring}, which uses a comparably small feature vector to \citet{machajdik2010affective}, resulting however in a 5\% classification increase to state-of-the-art approaches at the time.

Deep neural networks in this domain provide less transparency to the process with which emotions and sentiment are classified compared to feature analysis.
The emotional content of an image can be decomposed in various ways.
Image databases with singular emotion labels, and adjective-noun pairs (ANP) have been used for the training of deep neural network classifiers \citep{chen2014deepsentibank, yang2018visual} with up to 200\% performance gains over support vector machine classifiers.
The artificial neural network approach to general image classification domains has exploded in popularity when compared to feature decomposition approaches used by \citet{machajdik2010affective}.
This is largely due to their learning ability, particularly in recognising "hierarchical representations" \citep{lipton2015critical} of image features; which provides greatly improved performance when compared with manually crafted metrics derived through image decomposition.
This hands-off approach to image feature vector decomposition has further lent itself to techniques such as transfer learning.

Given the large amount of work conducted into image object classification, producing neural networks such as AlexNet, ResNet, and Inception, transfer learning has been heavily relied upon in various domains of classification and synthesis of images \citep{pan2009survey}.
Each of these pre-trained image classifiers can be used in other domains of machine learning by omitting the last layer of their architectures, which represent the object classification layer, and feed the previous feature layer into a new architecture.
In doing so, the information held within the neural network can be repurposed, and the learnings reused in other domains of application.
This allows 

Techniques used by image emotion recognition have been used extensively in domains such as image sentiment analysis \citep{you2015robust}, and image-to-text synthesis \citep{vinyals2015show}.
The subjectivity and human-dependent nature of emotion has naturally resulted in only small datasets of emotion-labelled images.
Through transfer learning, pre-trained image object classifiers can and have been used in the domain of image emotion \citep{kim2018building, wangarttalk}.

The use of continuous emotion representations, particularly relating to the circumplex model have been explored in image emotion recognition tasks \citep{kim2018building, zhao2016predicting, zhao2017continuous}.
Regression models produced to predict the valence-arousal (VA) values of given images have shown high accuracy on various datasets.
In leveraging pre-trained image classification networks through transfer learning, even smaller datasets (10,000 images) can have high accuracy classification results \citep{kim2018building}.
Recent datasets produced for image emotion recognition have used the valence-arousal-dominance model due to its continuity \citep{zhao2016predicting}.
While categorical classification is more easily verified by humans, training predictive models with data that has an element of noise and uncertainty benefits from both continuity and volume.
This is a key advantage of the circumplex model as applied by \citet{kim2018building} and \citet{zhao2016predicting} in image emotion recognition over categorical classification.


\section{Computational image synthesis}

Computer-generated images (CGI), has been a widely used in the film industry, as well as in countless other domains such as gaming, simulation, and art.
CGI has, for the most part, involved human interaction, and human-controlled image and model generation.
However generative systems and methods for both supervised, and unsupervised image synthesis have evolved in recent years with the increased use of evolutionary algorithms, and in more recent times, neural networks.
While extensive research has been conducted into the generation of visually aesthetic images, applications of image synthesis extend to the synthesis of text to describe a given image \citep{mathews2016senticap}, as well as the generation of an image according to a target caption \citep{reed2016generative,zhang2017stackgan}.


\subsection{Evolutionary computing}

Some of the first methods for image generation focused on the synthesis of visually appealing images.
While often using human-in-the-loop systems, visually striking and aesthetic images were the goal of methods introduced by \citet{sims} and \citet{nevar} involving evolutionary techniques.
\textit{NEvAr} \citep{nevar} was one of the first such image synthesis systems able to produce greatly impressive images through evolutionary techniques.
Evolutionary art leveraged methods introduced and exemplified by \citet{sims}, producing images such as those shown in Figure \ref{fig:sims}.
\citet{sims} proposed using \textit{Lisp} expressions for genotype definitions, which map a coordinate (x, y) into a grayscale or RGB value.
This genotype representation leveraged extensive research done into the use of evolutionary computing for optimisation problems.
This genotype expression has been used in numerous further research of both supervised and unsupervised image synthesis through evolutionary techniques \citep{nevar, sims, den2011evolving, distributed-evolutionary-art, aesthetic-measures}.
However the way in which NEvAr and Sims evolved images involved the manual process of selecting individuals in the population they deemed to be of higher fitness than the rest.

\begin{figure}[h!]
	\includegraphics[width=\textwidth]{images/sims-interactive-image-generation.png}
	\caption{Images generated through the process of interactive evolution introduced by \citet{sims}}
	\label{fig:sims}
\end{figure}

Evolutionary computing techniques for image synthesis have been able to produce increasingly interesting and appealing imagery and artwork.
While the genotype representation introduced by \citet{sims} was further used by countless bodies of research, there exist countless other unique methods for image synthesis.
Line-drawing as exemplified by \citet{niche-reproduction} has built upon evolutionary techniques, using a collection of individuals interacting in real-time to synthesis artwork.
In traditional well-mixed \citep{sims} or distributed population \citep{distributed-evolutionary-art} each individual represents an image, or a model for generating one; however the line-drawing model of image synthesis acts like a swarm of individuals, each drawing on a canvas as they move through the space.
This real-time evolutionary image drawing mechanism, while producing visually interesting images, has not been explored with regard to its use in other application domains.


\subsection{Measures of aesthetics}

Despite the slow nature of the interactive process, \citet{sims} and \citet{nevar} were able to produce images with visually striking characteristics.
\citet{aesthetic-measures} investigated measures of aesthetics for fitness evaluation in artificially evolving images.
This research primarily used observations by \citet{ralph-bell-curve}, that the distribution of colour gradients in fine art tend toward normal.
While the images produced through this method did not meet the level of intricacy and detail as the results of \citet{sims} or \citet{nevar}, it represented a self-contained system able to generate appealing imagery without human interaction.

Measures of aesthetics have been explored and multiple have been derived using information about fine art \citep{ralph-bell-curve}, and others measuring levels of symmetry, and even complexity measures according to image compression ratios \citep{den2010using}.
Work by \citet{den2014investigating} performed a comparison of seven measures of aesthetics, comparing even some of the most popular metrics such as the Ralph bell curve \citep{ralph-bell-curve}.
The primary finding of this work showed that the visual styles and nature of the images depended heavily on the given aesthetic metric used to determine fitness.
Particularly given that the process by which images are generated is an optimisation problem, maximising the aesthetic value as measured by the given metric, the metric used has an immense impact on the resulting styles.
However it was found that various pairs of metrics were correlated, resulting in images with highly similar characteristics when using one metric or the other.
\citet{den2014investigating} also further explored the multi-objective optimisation problem of image synthesis when using combinations of aesthetic measures.
The multi-objective optimisation variant of this research showed an increased aesthetic appeal of the images produced particularly with non-correlated metrics.


\subsection{Quality-diverse algorithms}

Recent work by \citet{nguyen2015innovation} and \citet{nguyen2015deep} investigated the use of quality-diverse algorithms for image generation particularly to better understand the patterns learned by deep neural network image classifiers.
Quality-diverse (QD) evolutionary algorithms such as Multi-dimensional Archive of Phenotypic Elites (MAP-Elites) \citep{mouret2015illuminating} and Novelty Search \citep{lehman2008exploiting, lehman2011abandoning} have been developed to address the need for a high quality, yet diverse solution space in related optimisation domains.
The type of problems QD algorithms aim to address include primarily those in which a multitude of solutions exist within a multi-objective space, however the degree to which each objective is desired may vary.
Thus algorithms such as MAP-Elites aims to maximise a given fitness function, while maintaining an N-dimensional feature space, where each dimensional represents the feature-specific fitness of a solution.
The MAP-Elites algorithm results in not just a single or set of high fitness solutions, but a collection of such high fitness solutions spatially distributed over the desired feature space.

The use of QD algorithms has shown great promise in its efficiency and accuracy on a number of hard optimisation problems \citep{pugh2016quality} such as maze navigation \citep{lehman2011abandoning}.
\citet{nguyen2015deep} and \citet{nguyen2015innovation} use MAP-Elites in conjunction with a pre-trained deep neural network (DNN) image classifier; assigning individual image fitness according to the accuracy with which it is classified.
Using the MAP-Elites framework in this context, each dimension of the feature-space represents a classification label, and as such the generated images allow the exploration of label representative patterns and shapes learnt by the classifier.
\citet{nguyen2015deep} leverages such an architecture to show the shallowness with which an image classifier recognises images.
Assigning the label of \textit{school bus} to alternating yellow and black lines is a prime example of the way such a network has learnt to differentiate one class from the others.
Thus enabling exploration into the inner workings of the DNN classifier by uncovering features that maximise the separation of one label to another.
In contrast, \citet{nguyen2015innovation} uses the same architecture to explore the novelty-driven evolutionary path taken by generated images and the potential for such a system in the field of content synthesis.
While the conclusions derived from \citet{nguyen2015innovation} and \citet{nguyen2015deep} contrast greatly, the quality-diverse generative method used to understand the visual components learnt by the classifier show such an architecture's exploratory abilities.
This technique for understanding the patterns learnt by such a classifier has not been explored in the context of regression.


\subsection{Generative adversarial neural networks}

Neural networks, while having been applied and researched extensively with regard to prediction and classification, as discussed in Section \ref{sec:neural-network-overview}, have recently shown exceedingly interesting and even visually realistic results with the generative adversarial architecture.
Their use is however limited with regard to image synthesis using more traditional feed-forward network architectures due to the difficulty in converting such networks into higher level image generators.
Limited work has been performed using image classifiers in conjunction with evolutionary computing techniques by \citet{nguyen2015innovation} as discussed previously.

The introduction of the generative adversarial network architecture (GAN) by \citet{GAN} allowed the process of image generation to depend only on collecting a sufficiently large dataset.
Common GAN application has involved the generation of realistic images, including work by \citet{bao2017cvae} where images have been synthesised to fine-detailed target labels such as bird species' and actors.
\citet{zhang2017stackgan} and \citet{reed2016generative} have recently explored text to image synthesis, in which detailed descriptions of birds and flowers have been converted into photo-realistic images using the GAN model.
Such an architecture has been applied to the area of art synthesis by \citet{tan2017artgan} in which images were generated according to a target genre and artist.
Learning from a dataset of countless artworks in various categories, styles, and artists, it was hugely successful in generating images that were stylistically similar to existing art of the target artist/genre.
Due to the competitive relationship of the generator and discriminator networks, the patterns learned by the discriminator propagate through the generator network.
The discriminator network of the GAN architecture aims to learn patterns and styles from the dataset on which it is trained in order to discriminate between the generated and existing images.
As a result, the images generated tend to resemble closely those in the training dataset, an advantage when similarity and realism to existing data is desired, and a detriment when generative creativity is a target attribute.


\section{Affective content synthesis}

The application of generative systems in the domain of affective computing, particularly with regard to emotion and content synthesis, is limited.
Sentiment-driven examples of generative systems include image captioning according to target sentiment \citep{mathews2016senticap}.
The task of describing an image was extended from a traditional GAN approach through the addition of a sentiment target input.
The method used to train such a generative system involved the conditional GAN architecture as described by \citet{gauthier2014conditional}.
A similar technique was used in style-driven image captioning (factual, romantic, humorous) in combination with a long short-term memory (LSTM) neural network model \citet{gan2017stylenet}.
In the context of image-to-image synthesis, \textit{emotion transfer} was explored by \citet{ali2017emotional} which involved the transformation of an image's colour and style with the aim of altering its conveyed emotion.


\section{Conclusion}

\bibliographystyle{apalike}
\bibliography{references}

\end{document}
