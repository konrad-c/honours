\documentclass[10pt,a4paper]{article}
\usepackage[utf8]{inputenc}
\usepackage{amsmath}
\usepackage{amsfonts}
\usepackage{amssymb}
\newcommand\tab[1][1cm]{\hspace*{#1}}
\usepackage{graphicx}
\usepackage{scrextend}
\usepackage{wrapfig}

\author{Konrad Cybulski}
\title{TITLE}

\begin{document}
	
\begin{titlepage}
	\begin{center}
		\vspace*{1cm}
		
		\LARGE
		\textbf{TITLE}
		
		\vspace{2cm}
		\Large
		
		\textbf{Konrad Cybulski}
		
		\vfill
		
		Honours Research Notes
		
		\vspace{0.8cm}
		
		\includegraphics[width=0.4\textwidth]{images/monash_emblem.jpg}
		
		\large
		Faculty of Information Technology\\
		Monash University\\
		Australia\\
		\today
		
	\end{center}
\end{titlepage}

\pagebreak
\tableofcontents
\pagebreak
\section{Research Notes}
There exist methods for manipulating graph topologies such that the resulting evolutionary system increases the likelihood with which high fitness individuals fixate \cite{graph amplifiersr, birth-death amplifiers}.
This process involves numerous system factors: edge weighting, birth-death initialisation, 
\\\\
The main paper on which I am aiming to base my research on is that by Pavlogiannis, Tkadlec, Chatterjee \& Nowak \cite{graph amplifiers}, which among other things, shows evidence of increasing fixation probabilities of advantageous mutants by structuring graph topologies according to a simple heuristic.
This involves logically separating a directed graph topology into two parts: a hub, and branches.
By doing so, and increasing weights directed towards the hub of a graph, the topology is able to maximise migration activitiy toward a centralised hub, and thus increase the likelihood of fixation for advantageous mutants.
\\\\

\section{Papers/Books}
\subsection{\cite{graph amplifiers} Creating arbitrarily strong amplifiers of natural selection on graphs}
\\\\
This paper explores one of the fundamental ideas upon which a large portion of the proposed research is based.
Graph topology has been shown to have a large impact on the rate of fixation for advantageous mutants \cite{graph amplifiers, birth-death amplifiers, cooperation on graphs}.
Numerous graph structures have been found that increase the likelihood of fixation above levels expected for a well-mixed population \cite{evolution on graphs, birth-death amplifiers}; these are known as amplifiers of selection.
While other topologies are known to decrease this likelihood, suppressors of selection.
Properties such as distribution of edge weights, edge directionaility \cite{birth-death amplifiers}, and self-loops are also shown to greatly impact fixation probabilities.
\\\\
As shown by Pavlogiannis, Tkadlec, Chatterjee, \& Nowak \cite{graph amplifiers}, not only can fixation probabilities be amplified by tuning the properties of a graph topology, but can be constructed arbitrarily.
This construction process involves adjusting edge weights to create two sections: a central hub, a series of branches.
The hub is constructed such that the nodes within it are tightly coupled, thus increasing the likelihood an advantageous strategy will fixate.
Then propagating through the branches, fixating on the entire graph.


\pagebreak
\section{Research Ideas}
With a method for constructing graphs that are arbitrary amplfiers of selection, my primary aim is to leverage this ability to create an evolutionary algorithm that supports generative agents with high fitness.
\\\\
Use methods discussed in \textit{Evolutionary mixed games in structured populations: Cooperation and the benefits of heterogeneity}.
By randomly selecting the game in which agents partake (snow drift, prisoner's dilemma), there is a proportional increase in population heterogeneity which increases cooperation.
In terms of a generative evolutionary algorithm, by using multiple fitness functions, randomly chosen, tested against all, or simply a subset of fitness functions, to determine overall fitness of an individual.

\pagebreak
\section{Log}

\subsection{22 Jan 2019}

I met with Jon McCormack yesterday to discuss potential project ideas and where to begin with my research.
Before the meeting I was able to

\section{Example Citations}
\cite{novak indirect reciprocity}
\cite{leading eight}
\cite{leading eight,novak indirect reciprocity,social norms small scale society}

\pagebreak

\begin{thebibliography}{9}

\bibitem{graph amplifiers}
Pavlogiannis, A., Tkadlec, J., Chatterjee, K., \& Nowak, M. A. (2018). Construction of arbitrarily strong amplifiers of natural selection using evolutionary graph theory. \textit{Communications Biology, 1}(1), 71.

\bibitem{evolution on graphs}
Lieberman, E., Hauert, C., \& Nowak, M. A. (2005). Evolutionary dynamics on graphs. \textit{Nature, 433}(7023), 312.
	
\bibitem{novak indirect reciprocity}
Martin A. Novak, Karl Sigmund, (2005) 
\textit{Evolution of indirect reciprocity}. 
Nature.

\bibitem{}
Amaral, M. A., Wardil, L., Perc, M., \& da Silva, J. K. (2016). Evolutionary mixed games in structured populations: Cooperation and the benefits of heterogeneity. 
\textit{Physical Review E, 93}(4), 042304.

\bibitem{cooperation on graphs}
Ohtsuki, H., Hauert, C., Lieberman, E., \& Nowak, M. A. (2006). A simple rule for the evolution of cooperation on graphs and social networks. \textit{Nature}, 441(7092), 502.

\bibitem{social norms small scale society} 
Fernando P. Santos, Francisco C. Santos, Jorge M. Pacheco, (2016) 
Social Norms of Cooperation in Small-Scale Societies.
\textit{PLOS Computational Biology}.

\bibitem{birth-death amplifiers}
Hindersin, L., \& Traulsen, A. (2015). Most undirected random graphs are amplifiers of selection for birth-death dynamics, but suppressors of selection for death-birth dynamics.
\textit{PLoS computational biology, 11}(11), e1004437.

\bibitem{leading eight}
Hisashi Ohtsuki, Yoh Iwasa, (2005) 
The leading eight: Social norms that can maintain cooperation by indirect reciprocity.
\textit{Journal of Theoretical Biology}.


\end{thebibliography}
\end{document}